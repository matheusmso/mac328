%-------------------------------------------------------------------------------
%	PACKAGES AND DOCUMENT CONFIGURATIONS
%-------------------------------------------------------------------------------

\documentclass[a4paper]{article}
\usepackage[left=3cm, right=3cm, bottom=2cm]{geometry}

\usepackage{graphicx} % Required for the inclusion of images
\usepackage{amsmath} % Required for some math elements
\usepackage{amsfonts}
\usepackage{amssymb}
\usepackage{amsthm} % Required for theorems
\usepackage[utf8]{inputenc}
\usepackage[usenames,dvipsnames]{color} % Required for custom colors
\usepackage{listings} % Required for insertion of code
\usepackage{courier} % Required for the courier font
\usepackage{enumitem}
\usepackage{titling}

\setlength{\droptitle}{-10em} % Moves title up
\setlength\parindent{0pt} % Removes all indentation from paragraphs
\linespread{1.6} % Double spacing

%--- For theorems/lemmas/etc ---
\newtheoremstyle{style}
{}
{}
{}
{}
{\bfseries}
{. }
{ }
{\thmname{#1}\thmnumber{ #2}\thmnote{ (#3)}}

\renewcommand*{\proofname}{Dem}                

\theoremstyle{style}
\newtheorem{thm}{Teorema}
\newtheorem{prop}{Proposição}
\newtheorem{lemma}{Lema}
\newtheorem*{defn}{Definição}
\newtheorem*{cor}{Corolário}
\theoremstyle{proof}

%-------------------------------------------------------------------------------
%	MACROS
%-------------------------------------------------------------------------------

%--- Sets of numbers ---
\newcommand{\N}{\mathbb{N}} % Natural set
\newcommand{\Z}{\mathbb{Z}} % Integer set
\newcommand{\Q}{\mathbb{Q}} % Rationals set
\newcommand{\R}{\mathbb{R}} % Reals set

%--- Fuck varepsilon ---
\newcommand{\eps}{\varepsilon}

%-------------------------------------------------------------------------------
%	DOCUMENT INFORMATION
%-------------------------------------------------------------------------------

\title{Lista 8 - MAC0328} % Title
\author{Matheus de Mello Santos Oliveira - 8642821} % Author name
\date{} % Date for the report
\begin{document}
\maketitle % Insert the title, author and date

%-------------------------------------------------------------------------------
%	WRITTEN STUFF
%-------------------------------------------------------------------------------

\section{Implementação}
Queremos implementar a função \texttt{augmentMatching()} usando busca em profundidade.
A implementação desta se parece muito com a implementação usando busca em largura,
basicamente sempre que a BFS adicionava alguém a fila, a DFS executa uma nova chamada.
O seguinte programa recebe um grafo bipartido definido no arquivo in e imprime o
tamanho de seu emparelhamento máximo.
\section{Uso do programa}
O programa recebe como entrada um arquivo que descreva um grafo da seguinte forma:
na primeira linha $V$ e $A$, nas próximas $A$ linhas dois inteiros representando
as arestas do grafo.
\\Para compilar:
\\\$ make ep8
\\Para executar o código:
\\\$ ./ep8 $<$ in
\\Onde in é um arquivo da forma descrita acima.

\end{document}

