%-------------------------------------------------------------------------------
%	PACKAGES AND DOCUMENT CONFIGURATIONS
%-------------------------------------------------------------------------------

\documentclass[a4paper]{article}
\usepackage[left=3cm, right=3cm, bottom=2cm]{geometry}

\usepackage{graphicx} % Required for the inclusion of images
\usepackage{amsmath} % Required for some math elements
\usepackage{amsfonts}
\usepackage{amssymb}
\usepackage{amsthm} % Required for theorems
\usepackage[utf8]{inputenc}
\usepackage[usenames,dvipsnames]{color} % Required for custom colors
\usepackage{listings} % Required for insertion of code
\usepackage{courier} % Required for the courier font
\usepackage{enumitem}
\usepackage{titling}

\setlength{\droptitle}{-10em} % Moves title up
\setlength\parindent{0pt} % Removes all indentation from paragraphs
\linespread{1.6} % Double spacing

%--- For theorems/lemmas/etc ---
\newtheoremstyle{style}
{}
{}
{}
{}
{\bfseries}
{. }
{ }
{\thmname{#1}\thmnumber{ #2}\thmnote{ (#3)}}

\renewcommand*{\proofname}{Dem}                

\theoremstyle{style}
\newtheorem{thm}{Teorema}
\newtheorem{prop}{Proposição}
\newtheorem{lemma}{Lema}
\newtheorem*{defn}{Definição}
\newtheorem*{cor}{Corolário}
\theoremstyle{proof}

%-------------------------------------------------------------------------------
%	MACROS
%-------------------------------------------------------------------------------

%--- Sets of numbers ---
\newcommand{\N}{\mathbb{N}} % Natural set
\newcommand{\Z}{\mathbb{Z}} % Integer set
\newcommand{\Q}{\mathbb{Q}} % Rationals set
\newcommand{\R}{\mathbb{R}} % Reals set

%--- Fuck varepsilon ---
\newcommand{\eps}{\varepsilon}

%-------------------------------------------------------------------------------
%	DOCUMENT INFORMATION
%-------------------------------------------------------------------------------

\title{Lista 7 - MAC0328} % Title
\author{Matheus de Mello Santos Oliveira - 8642821} % Author name
\date{} % Date for the report
\begin{document}
\maketitle % Insert the title, author and date

%-------------------------------------------------------------------------------
%	WRITTEN STUFF
%-------------------------------------------------------------------------------

\section{Implementação}
Queremos encontrar uma k coloração em um grafo. Para isso realizamos um
backtrack que realiza uma busca completa em todas as possíveis colorações
e verifica se alguma delas é valida, isto é, para alguma dessas colorações
todos as arestas possuem vértice de início e fim de cores diferentes.
Uma redução no espaço de busca está implícita na ideia de um vértice não 
deve ser colorido da mesma cor de algum de seus vizinhos. Podemos limitar
superiormente essa função com $O(k^n)$, mas podemos perceber que o fato do
backtrack não permitir colorações inválidas, varias dessas opções não serão
testadas, o que torna esse upper bound bastante folgado.
\section{Experimentos}
Realizamos testes para $V = \{5, 10, 35\}$ e $E = \{V, 2V, 5V\}$ e 
$k = \{2, 3, 5, 8\}$. Todos os testes foram repetidos 10 vezes.
O número de vezes em que k cores foram suficientes para pintar o grafo.
Linhas são números de vértices e colunas são número de arestas.
Para $k = 2$
\begin{tabular} { | c | c | c | c | }
    \hline
       & V & 2V & 5V  \\
    \hline
    5  & 8 & 0  &  0  \\
    \hline
    10 & 3 & 0  &  0  \\
    \hline
    35 & 0 & 0  &  0  \\
    \hline
\end{tabular}
\\Para $k = 3$
\begin{tabular} { | c | c | c | c | }
    \hline
       & V & 2V & 5V  \\
    \hline
    5  & 10 & 0   &  0  \\
    \hline
    10 & 10 & 10  &  0  \\
    \hline
    35 & 10 & 5   &  0  \\
    \hline
\end{tabular}
\\Para $k = 5$
\begin{tabular} { | c | c | c | c | }
    \hline
       & V & 2V & 5V  \\
    \hline
    5  & 10 & 10  & 10  \\
    \hline
    10 & 10 & 10  &  0  \\
    \hline
    35 & 10 & 10  &  9  \\
    \hline
\end{tabular}
\\Para $k = 8$
\begin{tabular} { | c | c | c | c | }
    \hline
       & V & 2V & 5V  \\
    \hline
    5  & 10 & 10  & 10  \\
    \hline
    10 & 10 & 10  &  0  \\
    \hline
    35 & 10 & 10  &  10 \\
    \hline
\end{tabular}
\\Importante  notar que em geral a heurística gulosa retorna um resultado similar, porém
com uma fração do custo computacional.
\section{Uso do programa}
O programa recebe como argumento da linha de comando o número de vértices
o número de arestas e a quantidade de cores que desejamos usar para 
colorir o grafo.
\\Para compilar:
\\\$ make
\\Para executar o código:
\\\$ ./ep7 [V] [A] [k]
\\Onde $V$ é o número de vértices e $A$ é o número de arestas e $k$ é o 
número de cores disponíveis.

\end{document}

